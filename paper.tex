% !TeX program = xelatex 
%% 
%% Copyright 2019-2021 Elsevier Ltd
%% 
%% This file is part of the 'CAS Bundle'.
%% --------------------------------------
%% 
%% It may be distributed under the conditions of the LaTeX Project Public
%% License, either version 1.2 of this license or (at your option) any
%% later version.  The latest version of this license is in
%%    http://www.latex-project.org/lppl.txt
%% and version 1.2 or later is part of all distributions of LaTeX
%% version 1999/12/01 or later.
%% 
%% The list of all files belonging to the 'CAS Bundle' is
%% given in the file `manifest.txt'.
%% 
%% Template article for cas-dc documentclass for 
%% double column output.

\documentclass[a4paper,fleqn,twocolumn]{cas-dc}



% If the frontmatter runs over more than one page
% use the longmktitle option.

%\documentclass[a4paper,fleqn,longmktitle]{cas-dc}

%\usepackage[numbers]{natbib}
%\usepackage[authoryear]{natbib}
\usepackage[authoryear,longnamesfirst]{natbib}

%\usepackage[numbers]{natbib}
%\usepackage[authoryear]{natbib}
%\usepackage[authoryear,longnamesfirst]{natbib}
\usepackage{dsfont}
\usepackage{array}
\usepackage{float}
\usepackage{placeins}
\usepackage{makecell} % add by shuyun
%\usepackage[utf8]{inputenc}
\usepackage{lineno,hyperref}
\usepackage{amsmath}
\usepackage{graphicx}
\usepackage[small]{caption}
\usepackage{subcaption}
\usepackage{array}
\usepackage{multirow}
\usepackage{multicol}
\usepackage{diagbox}
\usepackage{mathrsfs}
\usepackage{color}
\usepackage{booktabs}
\usepackage{soul}
\usepackage{mathspec}
\usepackage{xeCJK}
%

%\setCJKmainfont{STKaiti}

%\setCJKmainfont[Mapping=tex-text]{STKaiti}
\usepackage[ruled, linesnumbered, vlined]{algorithm2e}

%\linenumbers
%%%Author macros
\def\tsc#1{\csdef{#1}{\textsc{\lowercase{#1}}\xspace}}
\tsc{WGM}
\tsc{QE}


\newcommand{\1}[1]{\mathds{1}\left[#1\right]}
\renewcommand{\floatpagefraction}{0.8}

\newcommand{\secref}[1]{Section \ref{#1}}
\newcommand{\figref}[1]{Figure \ref{#1}}
\newcommand{\eqnref}[1]{Eq. (\ref{#1})}
\newcommand{\exref}[1]{Example \ref{#1}}
\newcommand{\algoref}[1]{Algorithm \ref{#1}}
\newcommand{\tableref}[1]{Table \ref{#1}}
\newcommand{\socvec}{SocVec}
\newcommand{\argmin}{\operatornamewithlimits{argmin}}
\newcommand{\argmax}{\operatornamewithlimits{argmax}}
\newtheorem{example}{Example}
\newtheorem{lemma}{Lemma}
\newtheorem{definition}{Definition}
\newcommand{\cut}[1]{}
\newcommand{\ZY}[1]{\textcolor{red}{Zhiyi: #1}}

\begin{document}
%	\begin{multicols}{2}
\let\WriteBookmarks\relax
\def\floatpagepagefraction{1}
\def\textpagefraction{.001}

% Short title
\shorttitle{Empowering Multi-Span Question Answering with Expansive Information Injection using Large Language Models}    

% Short author
\shortauthors{Z. Luo et al.}  

% Main title of the paper
\title{Empowering Multi-Span Question Answering with Expansive Information Injection using Large Language Models}  

\author{Zhiyi Luo}[type=editor,auid=000,bioid=1,orcid=0000-0002-2206-1926]
\ead{luozhiyi@zstu.edu.cn}
\ead[url]{http://zhiyiluo.site}
%\credit{Conceptualization of this study, Methodology, Software}

\author{Yingying Zhang}
\ead{272831920@qq.com}

\author{Ying Zhao}
\ead{308956149@qq.com}

\author{Shuyun Luo}
\cormark[1]
\cortext[mycorrespondingauthor]{Corresponding author}
\ead{shuyunluo@zstu.edu.cn}
%\cormark[1]
%\cortext[mycorrespondingauthor]{Corresponding author}

\author{Wentao Lv}
\ead{alvinlwt@zstu.edu.cn}

\address[mymainaddress]{School of Computer Science and Technology and the Key Laboratory of Intelligent Textile and Flexible Interconnection of Zhejiang Province, Zhejiang Sci-Tech University}
\address[mysecondaryaddress]{No. 928, No. 2 street, Baiyang street, Qiantang New District, Hangzhou 310018, China}


%\Title{Empowering Multi-Span Question Answering with Expansive Information Injection using Large Language Models}
%
%% MDPI internal command: Title for citation in the left column
%\TitleCitation{Title}
%
%% Author Orchid ID: enter ID or remove command
%\newcommand{\orcidauthorA}{0000-0002-2206-1926} % Add \orcidA{} behind the author's name
%%\newcommand{\orcidauthorB}{0000-0000-0000-000X} % Add \orcidB{} behind the author's name
%
%
%% Authors, for the paper (add full first names)
%\Author{Zhiyi Luo $^{1}$\orcidA{}, Yingying Zhang$^{1}$ and Shuyun Luo $^{1,}$*}
%
%%\longauthorlist{yes}
%
%% MDPI internal command: Authors, for metadata in PDF
%\AuthorNames{Zhiyi Luo, Yingying Zhang and Shuyun Luo}
%
%% MDPI internal command: Authors, for citation in the left column
%\AuthorCitation{Luo, Z.; Zhang, Y.; Luo, S.}
%% If this is a Chicago style journal: Lastname, Firstname, Firstname Lastname, and Firstname Lastname.
%
%% Affiliations / Addresses (Add [1] after \address if there is only one affiliation.)
%\address{%
%$^{1}$ \quad School of Computer Science and Technology and the Key Laboratory of Intelligent Textile and Flexible Interconnection of Zhejiang Province, Zhejiang Sci-Tech University, Hangzhou, China; luozhiyi@zstu.edu.cn
%%$^{2}$ \quad Affiliation 2; e-mail@e-mail.com
%}
%
%% Contact information of the corresponding author
%\corres{Correspondence: shuyunluo@zstu.edu.cn;}

% Current address and/or shared authorship
%\firstnote{Current address: Affiliation 3.} 
%\secondnote{These authors contributed equally to this work.}
% The commands \thirdnote{} till \eighthnote{} are available for further notes

%\simplesumm{} % Simple summary

%\conference{} % An extended version of a conference paper

% % Here goes the abstract
\maketitle


\begin{abstract}
	Multi-span question answering has gained prominence as it aligns more closely with real-world user requirements compared to single-span question answering. The utilization of pretrained language models has shown promise in improving multi-span question answering, particularly for factoid questions that necessitate entity-based answers. However, existing methods tend to overlook critical information regarding answer span boundaries, resulting in limited accuracy when generating descriptive answers. To address this limitation, we propose TOAST, a novel joint learning framework specialized in token-based neighboring transitions that capture answer span boundaries through adjacent word relations. Our approach extracts high-quality multi-span answers and is general-purpose, applicable to both alphabet languages like English and logographic languages like Chinese. Furthermore, we introduce CLEAN, a comprehensive open-domain Chinese multi-span question answering dataset, which includes a substantial number of descriptive questions. 
	Extensive experiments demonstrate the superior performance of TOAST over previous top-performing QA models in terms of both EM F1 and overlapped F1 scores. Specifically, the TOAST models, leveraging  $\text{BERT}_{base}$ and $\text{RoBERTa}_{base}$, achieve substantial  improvements in EM F1 scores, with increments of 3.03/2.13, 4.82/3.73, and 16.26/11.53, across three publicly available datasets, respectively. 
\end{abstract}

% Keywords
\begin{keywords}
	Reading comprehension \sep Multi-span question answering \sep  Pretrained language models \sep Multitask learning  \sep Chinese datasets 
\end{keywords}



%%%%%%%%%%%%%%%%%%%%%%%%%%%%%%%%%%%%%%%%%%
\section{Introduction}
\label{sec:intro}

Let's cite a paper~\citep{DBLP:conf/aaai/PangLGXSC19}.

In summary, the main contributions in this paper are as follows:
\begin{itemize}
	\item We employs a automatic data augmentation framework using Large Language Model \((LLM)\) as a knowledge source and a extra content supplement to linearize relevant information and possible continuation from LLM as texts, then inject them into original contexts. 
	\item We develop a series of prompt templates designed for interacting with ChatGPT to acquire comprehensive explanations of numerous entities. These templates ensure that the formats of the responses provided by ChatGPT are highly parseable and well-structured.
	
\end{itemize}

%The introduction should briefly place the study in a broad context and highlight why it is important. It should define the purpose of the work and its significance. The current state of the research field should be reviewed carefully and key publications cited. Please highlight controversial and diverging hypotheses when necessary. Finally, briefly mention the main aim of the work and highlight the principal conclusions. As far as possible, please keep the introduction comprehensible to scientists outside your particular field of research. Citing a journal paper \cite{ref-journal}. Now citing a book reference \cite{ref-book1,ref-book2} or other reference types \cite{ref-unpublish,ref-communication,ref-proceeding}. Please use the command \citep{ref-thesis,ref-url} for the following MDPI journals, which use author--date citation: Administrative Sciences, Arts, Econometrics, Economies, Genealogy, Humanities, IJFS, Journal of Intelligence, Journalism and Media, JRFM, Languages, Laws, Religions, Risks, Social Sciences, Literature.
%%%%%%%%%%%%%%%%%%%%%%%%%%%%%%%%%%%%%%%%%%

<<<<<<< HEAD
<<<<<<< HEAD

\section{Related Work}
\label{sec:related}
In this section, we briefly summarize previous work on how Large Language Models (LLMs) such as the GPT series can be utilized as sources of knowledge and content generators to enhance and enrich text data in various application scenarios.
\subsection{Interaction with LLMs}
Large Language Models (LLMs) are powerful natural language processing models in the field of deep learning, 
characterized by their massive parameter scale and outstanding capabilities in understanding and generating natural language text. 
Recent research has highlighted the extraction of relevant knowledge from LLMs, especially in domains lacking proper coverage in knowledge bases~\citep{fang2021leveraging}. 
The application of large language models has expanded from the field of knowledge extraction to text data augmentation. 
GPT3Mix~\citep{yoo2021gpt3mix} has proposed a method that utilizes large language models to generate new text samples, thereby enhancing the performance of machine learning models.
AugGPT~\citep{dai2023auggpt} enhances natural language processing tasks in situations with limited data by interacting with ChatGPT to generate new textual data, further advancing text data augmentation.
In the realm of common sense reasoning,~\citep{liu2021generated,liu2022rainier} suggests that performance in common sense question answering can be effectively improved through prompt generation and reinforcement learning. 
Additionally, LLMs excel in generating contextual information, demonstrating the ability to create rich, 
relevant contexts based on given inputs, as validated in studies such as those by~\citep{yu2022generate}. 
The collaborative effect of iterative retrieval and generation further boosts the performance of LLMs~\citep{shao2023enhancing}. 
However, a limitation in current research is the insufficient interpretation of entities.Our study addresses this gap by interacting with ChatGPT, 
leveraging its powerful generative capabilities to design a series of prompt templates for obtaining more comprehensive entity interpretations.


In the research domain of question-answering systems, ~\citep{huang2023enhancing} improves the context learning ability of models in multi-span question-answering tasks through the introduction of answer feedback mechanisms. 
Comparisons between ChatGPT and traditional knowledge base question-answering models explore its potential as an alternative to traditional KBQA tasks~\citep{tan2023can}, 
investigating differences in providing single versus multiple answers in various contexts within reading comprehension tasks~\citep{zhang2023many}. 
Considering the limitations of LLMs in handling factual information, ~\citep{ren2023investigating} utilizes retrieval-enhanced techniques to enhance the model's understanding and response capabilities to fact-based queries.
Finally, ~\citep{wei2023zero} showcase a new approach to zero-shot information extraction through interaction with ChatGPT, 
while~\citep{liu2023once} explore the enhancement of content-based recommendation systems by combining open-source and closed-source language models. 
These studies not only demonstrate the innovation and advantages of LLMs across various domains but also provide profound insights into future research directions. 
However, the response structures generated by LLMs in these studies are not standardized. 
To address this, our prompt template design aims to ensure that the responses from ChatGPT are easily parsed and processed, enhancing the structured nature of the responses. 
Additionally, dealing with ambiguity and ambiguity is another challenge. 
Through the integration of LLM knowledge sources, our framework aims to provide clearer, more detailed information to better address semantic complexities.


\subsection{Neural Models for RC}
Research in reading comprehension grows rapidly, and many successful neural-based RC models have been proposed in this area. 
%As discussed in \secref{sec:related_dataset}, the extractive RC datasets support to cast RC as answer extraction task.
Typically, neural  models~\citep{DBLP:conf/aaai/PangLGXSC19,DBLP:conf/iclr/Wang017a,DBLP:conf/iclr/XiongZS17} for RC are composed of two components, a context encoder and an answer decoder. 
The context encoder is used to encode the information of questions, contexts and their interactions in-between. Then, the answer decoder aims to generate the answer texts based on outputs of the context encoder. 
To make the answer decoder compatible with the answer extraction task, 
Pointer Network~\citep{DBLP:conf/nips/VinyalsFJ15} model has been adopted to
% tackle the problem that 
%generates answer texts from contexts. 
copy tokens from the given contexts as answers~\citep{DBLP:conf/acl/KadlecSBK16,DBLP:conf/emnlp/TrischlerYYBSS16}.
\cite{DBLP:conf/iclr/Wang017a} proposed a boundary model, which utilized Pointer Network to predict the start and end indices for an answer span.
%\ZY{single token - single span - multispan}
%Boundary Model
\cite{DBLP:conf/iclr/SeoKFH17} proposed an alternative way for the implementation of answer decoder, that built neural position classifiers upon the encoder outputs, predicting the start and end indices of the answer span in the context.

% neural layer unifies the components of encoder and decoder together that 
Recently, the RC models upgrade the context encoder using pre-trained language models (PrLMs)~\citep{radford2018improving,kenton2019bert,liu2019roberta,lee2020biobert,gu2021domain} 
%such as GPT~\citep{radford2018improving}, BERT~\citep{kenton2019bert} and RoBERTa~\citep{liu2019roberta}
, benefiting from the invention of Transformer~\citep{vaswani2017attention} blocks.
%Benefiting from nowadays pretrained language models (PrLMs) based on Transformer~\cite{vaswani2017attention} blocks, recent RC models upgrades the context encoder using PrLMs such as BERT. 
%the context encoder and answer decoder can be unified into a standard algorithm that utilize PrLMs such as BERT to encode inputs, then predict the start position and end position for the answer span.
\cite{DBLP:conf/naacl/DevlinCLT19}  proposed a standard extractive model for  single-span RC that utilizes BERT to encode inputs, then builds position classifiers to predict where the answer span starts and ends.
However, the answer decoder, whether implemented with Pointer Network or position classifiers, predicts start and end position independently, thus can not distinguish the different answer spans properly.
%Pang et al. proposed HAS-QA~\cite{} which built the conditional pointer network and aggregators to model multiple answer spans.
~\cite{DBLP:conf/emnlp/ZhuAJ0R20} proposed MultiCo which used a contextualized sentence selection method to capture the relevance among multiple sentence-based answer spans in order to form an answer with multiple sentences.
%However, such single-span extractive model is not suitable for multi-span RC task which can be formulated as multi-span extraction.
These models are not well adapted to multi-span RC which can be formulated as more flexible task of multi-span extraction where each span can be a word, phrase, sentence or any continuous string of text.

%the start position and end position of the answer span. 
%PrLMs unify the context encoder and answer decoder as a standard algorithm, that  
%\subsubsection{Single-span Models}
%~\citep{yoon2022,DBLP:journals/corr/abs-2302-01691} 
Extracting a variable number of spans from an input text can be commonly cast as a sequence tagging problem.
\cite{segal2020simple} proposed using a sequence tagging model for multi-span extraction, which predicts whether each token is part of an answer. \cite{yoon2022} employed a similar sequence tagging approach to address extractive question answering~\citep{DBLP:journals/bmcbi/NaseemDKK22}  in the biomedical domain.
\cite{li2022multispanqa} also adopted the tagging model architecture, integrating two sub-tasks: predicting the number of spans to extract and annotating the answer structure within their proposed dataset to capture global information.
ADRAV~\citep{hu2023biomedical} proposed a dynamic routing and answer voting method to further make full use of the hidden layer knowledge of pre-trained models.

More recently, SpanQualifier~\cite{huang2023qualifier} improves Multi-Span Reading Comprehension (MSRC) by explicitly representing spans and modeling interactions within and between them.
Iterative Extractor~\cite{zhang2023many} introduces a classification method for MSRC instances, prompting exploration of strategies to maximize different paradigms' advantages in capturing key information and modeling relationships between questions and context.
While these methods leverage powerful context encoders (PrLMs) with good multi-span answer extraction performance, they fall short in fully harnessing external knowledge, limiting their text comprehension abilities.
LIQUID~\cite{lee2023liquid} was designed to automatically generate list-style QA pairs from unlabeled corpora, using named entities from summarized text as candidate answers and incorporating synthetic data in the tagging model. However, its focus on list QA data narrows the scope of knowledge and lacks specificity, reducing efficiency.
In contrast, AUG significantly advances the field by efficiently enhancing model training. It achieves this by leveraging interactions with a large language model to acquire contextually relevant knowledge directly tied to the data, which can be seamlessly incorporated into the finetuning.


\subsection{Interaction with LLMs}
=======
<<<<<<< HEAD

\section{Related Work}
\label{sec:related}
In this section, we briefly summarize previous work on how Large Language Models (LLMs) such as the GPT series can be utilized as sources of knowledge and content generators to enhance and enrich text data in various application scenarios.
\subsection{Interaction with LLMs}
Large Language Models (LLMs) are powerful natural language processing models in the field of deep learning, 
characterized by their massive parameter scale and outstanding capabilities in understanding and generating natural language text. 
Recent research has highlighted the extraction of relevant knowledge from LLMs, especially in domains lacking proper coverage in knowledge bases~\citep{fang2021leveraging}. 
The application of large language models has expanded from the field of knowledge extraction to text data augmentation. 
GPT3Mix~\citep{yoo2021gpt3mix} has proposed a method that utilizes large language models to generate new text samples, thereby enhancing the performance of machine learning models.
AugGPT~\citep{dai2023auggpt} enhances natural language processing tasks in situations with limited data by interacting with ChatGPT to generate new textual data, further advancing text data augmentation.
In the realm of common sense reasoning,~\citep{liu2021generated,liu2022rainier} suggests that performance in common sense question answering can be effectively improved through prompt generation and reinforcement learning. 
Additionally, LLMs excel in generating contextual information, demonstrating the ability to create rich, 
relevant contexts based on given inputs, as validated in studies such as those by~\citep{yu2022generate}. 
The collaborative effect of iterative retrieval and generation further boosts the performance of LLMs~\citep{shao2023enhancing}. 
However, a limitation in current research is the insufficient interpretation of entities.Our study addresses this gap by interacting with ChatGPT, 
leveraging its powerful generative capabilities to design a series of prompt templates for obtaining more comprehensive entity interpretations.


In the research domain of question-answering systems, ~\citep{huang2023enhancing} improves the context learning ability of models in multi-span question-answering tasks through the introduction of answer feedback mechanisms. 
Comparisons between ChatGPT and traditional knowledge base question-answering models explore its potential as an alternative to traditional KBQA tasks~\citep{tan2023can}, 
investigating differences in providing single versus multiple answers in various contexts within reading comprehension tasks~\citep{zhang2023many}. 
Considering the limitations of LLMs in handling factual information, ~\citep{ren2023investigating} utilizes retrieval-enhanced techniques to enhance the model's understanding and response capabilities to fact-based queries.
Finally, ~\citep{wei2023zero} showcase a new approach to zero-shot information extraction through interaction with ChatGPT, 
while~\citep{liu2023once} explore the enhancement of content-based recommendation systems by combining open-source and closed-source language models. 
These studies not only demonstrate the innovation and advantages of LLMs across various domains but also provide profound insights into future research directions. 
However, the response structures generated by LLMs in these studies are not standardized. 
To address this, our prompt template design aims to ensure that the responses from ChatGPT are easily parsed and processed, enhancing the structured nature of the responses. 
Additionally, dealing with ambiguity and ambiguity is another challenge. 
Through the integration of LLM knowledge sources, our framework aims to provide clearer, more detailed information to better address semantic complexities.


\subsection{Neural Models for RC}
Research in reading comprehension grows rapidly, and many successful neural-based RC models have been proposed in this area. 
%As discussed in \secref{sec:related_dataset}, the extractive RC datasets support to cast RC as answer extraction task.
Typically, neural  models~\citep{DBLP:conf/aaai/PangLGXSC19,DBLP:conf/iclr/Wang017a,DBLP:conf/iclr/XiongZS17} for RC are composed of two components, a context encoder and an answer decoder. 
The context encoder is used to encode the information of questions, contexts and their interactions in-between. Then, the answer decoder aims to generate the answer texts based on outputs of the context encoder. 
To make the answer decoder compatible with the answer extraction task, 
Pointer Network~\citep{DBLP:conf/nips/VinyalsFJ15} model has been adopted to
% tackle the problem that 
%generates answer texts from contexts. 
copy tokens from the given contexts as answers~\citep{DBLP:conf/acl/KadlecSBK16,DBLP:conf/emnlp/TrischlerYYBSS16}.
\cite{DBLP:conf/iclr/Wang017a} proposed a boundary model, which utilized Pointer Network to predict the start and end indices for an answer span.
%\ZY{single token - single span - multispan}
%Boundary Model
\cite{DBLP:conf/iclr/SeoKFH17} proposed an alternative way for the implementation of answer decoder, that built neural position classifiers upon the encoder outputs, predicting the start and end indices of the answer span in the context.

% neural layer unifies the components of encoder and decoder together that 
Recently, the RC models upgrade the context encoder using pre-trained language models (PrLMs)~\citep{radford2018improving,kenton2019bert,liu2019roberta,lee2020biobert,gu2021domain} 
%such as GPT~\citep{radford2018improving}, BERT~\citep{kenton2019bert} and RoBERTa~\citep{liu2019roberta}
, benefiting from the invention of Transformer~\citep{vaswani2017attention} blocks.
%Benefiting from nowadays pretrained language models (PrLMs) based on Transformer~\cite{vaswani2017attention} blocks, recent RC models upgrades the context encoder using PrLMs such as BERT. 
%the context encoder and answer decoder can be unified into a standard algorithm that utilize PrLMs such as BERT to encode inputs, then predict the start position and end position for the answer span.
\cite{DBLP:conf/naacl/DevlinCLT19}  proposed a standard extractive model for  single-span RC that utilizes BERT to encode inputs, then builds position classifiers to predict where the answer span starts and ends.
However, the answer decoder, whether implemented with Pointer Network or position classifiers, predicts start and end position independently, thus can not distinguish the different answer spans properly.
%Pang et al. proposed HAS-QA~\cite{} which built the conditional pointer network and aggregators to model multiple answer spans.
~\cite{DBLP:conf/emnlp/ZhuAJ0R20} proposed MultiCo which used a contextualized sentence selection method to capture the relevance among multiple sentence-based answer spans in order to form an answer with multiple sentences.
%However, such single-span extractive model is not suitable for multi-span RC task which can be formulated as multi-span extraction.
These models are not well adapted to multi-span RC which can be formulated as more flexible task of multi-span extraction where each span can be a word, phrase, sentence or any continuous string of text.

%the start position and end position of the answer span. 
%PrLMs unify the context encoder and answer decoder as a standard algorithm, that  
%\subsubsection{Single-span Models}
%~\citep{yoon2022,DBLP:journals/corr/abs-2302-01691} 
Extracting a variable number of spans from an input text can be commonly cast as a sequence tagging problem.
\cite{segal2020simple} proposed using a sequence tagging model for multi-span extraction, which predicts whether each token is part of an answer. \cite{yoon2022} employed a similar sequence tagging approach to address extractive question answering~\citep{DBLP:journals/bmcbi/NaseemDKK22}  in the biomedical domain.
\cite{li2022multispanqa} also adopted the tagging model architecture, integrating two sub-tasks: predicting the number of spans to extract and annotating the answer structure within their proposed dataset to capture global information.
ADRAV~\citep{hu2023biomedical} proposed a dynamic routing and answer voting method to further make full use of the hidden layer knowledge of pre-trained models.

More recently, SpanQualifier~\cite{huang2023qualifier} improves Multi-Span Reading Comprehension (MSRC) by explicitly representing spans and modeling interactions within and between them.
Iterative Extractor~\cite{zhang2023many} introduces a classification method for MSRC instances, prompting exploration of strategies to maximize different paradigms' advantages in capturing key information and modeling relationships between questions and context.
While these methods leverage powerful context encoders (PrLMs) with good multi-span answer extraction performance, they fall short in fully harnessing external knowledge, limiting their text comprehension abilities.
LIQUID~\cite{lee2023liquid} was designed to automatically generate list-style QA pairs from unlabeled corpora, using named entities from summarized text as candidate answers and incorporating synthetic data in the tagging model. However, its focus on list QA data narrows the scope of knowledge and lacks specificity, reducing efficiency.
In contrast, AUG significantly advances the field by efficiently enhancing model training. It achieves this by leveraging interactions with a large language model to acquire contextually relevant knowledge directly tied to the data, which can be seamlessly incorporated into the finetuning.


\subsection{Interaction with LLMs}
>>>>>>> 14af3a787717720ea39cd58099013268fd5004a2

\section{Approach}
\subsection{Our Framework}
\begin{figure*}[h]
	\centering
	\includegraphics[width=18cm]{overview.png}
	\caption{An overview of our automatic information augmentation framework. \textbf{(a) Step 1}: Interact with ChatGPT to Get Auxiliary Information. \textbf{(b) Step 2}: Distilling the Information and Injecting Them into Context \textbf{(c) Step 3}: Input the Augmented Context into Tagging Model}
	\label{fig:overview}
\end{figure*}   

 Given a question $Q_i \in Q = \{Q_1,...,Q_n\}$ and a context $C_i \in C = \{C_0,...,C_n\}$, where $C_i$ contains m tokens $c_0,..,c_m$, the objective of multi-span question answering is to identify a set of answer spans $A_i = \{a_0,...,a_s\}$ within the context. Here, each answer span $a_j \in A_i$ is represented as $a_j = c_{s_l},...,c_{e_l}$, where $s_l$ and $e_l$ denote the start and end positions of the l-th answer span, respectively.
 Following the observation of~\cite{li2022multispanqa}, We adopt the BIO tagging scheme to mark answer spans in the context where words are tagged as either part of the answer(\textbf{B}egin, \textbf{I}nside) or not (\textbf{O}ther). Formally, BIO tagging scheme is represented by a tag set $\tau = \{B, I, O\}$.

 Intuitively, large language models (LLMs) can serve as supplementary sources of external knowledge, compensating for the restricted semantic comprehension and limited information perception inherent in pre-trained language models). 
 We then propose a novel \textbf{G}enerative \textbf{I}nformation \textbf{A}ugme\textbf{NT}ation framework (GIANT) for multi-span question answering.
 GIANT employs a plug-and-play strategy to integrate the knowledge from large language models into the input layer of tagging models, built upon pre-trained language models.
 
 
 The overview of GIANT is depicted as \figref{fig:overview}.
 This process comprises the following steps:
 \textbf{1)Prompting}: constructing instruction templates to leverage language models for generating diversified data, involving entity elucidation, entity relationships, content continuation, and summarization.
 \textbf{2)Generating}: the large language model generating new knowledge based on the designed prompts.
 \textbf{3)Updating}: filtering the generated data, and combine it with metadata in different forms.
 \textbf{4)Training}: employing the knowledge-enhanced data to train a tagging model.

 In the following section, we will explore how GIANT utilizes a large language model to generate knowledge. Subsequently, we will delve into the methods employed by GIANT to filter these knowledge, amalgamate them into meta-context, and orchestrate ensemble strategies to leverage these knowledge effectively.

\textcolor{red}{提示的模板放在哪里描述比较好?}
\label{sec:prompt_construction}
\begin{figure*}[h]
	\centering
	\includegraphics[width=14.5cm]{Prompts_construction.png}
	\caption{Prompt Templates and it's Construction}
	\label{fig:prompt_template}
\end{figure*}   

\subsection{LLMs as Knowledge Source}
 GIANT leverages a large language model as an external knowledge source, utilizing it to generate augmented data $K_i = \{E_i, S_i, R_i, F_i\}$ from multiple perspectives. Here, $E_i$ represents named entity elucidation, $R_i$ denotes entity relations, $S_i$ pertains to content summarization, and $F_i$ encompasses content continuation.

 Within these knowledge perspectives, named entity elucidation and entity relations provide factual knowledge, with GIANT synthesizing knowledge cues for the model by jointly injecting them into metadata.
 Summarization acts as a mechanism for information filtering, sieving and retaining entity interpretations and analyses of entity relationship, thereby ensuring augmentation efficiency. 
 Content continuation introduces relevant external knowledge by extending contextual content.
 
\subsubsection{LLMs as Content Summerizer}
 One of the pivotal factors in entity knowledge selection lies in ensuring that these entities exhibit semantic relevance within the specified context. The incorporation of disparate entity explanations and relationship analyses at the model's input layer may lead to the valueless augmentation of contexts, consequently impeding the model's learning.
 
 Utilizing the potent capabilities of large language models, we employ them to succinctly summarize the context $C_i$ into $S_i$, followed by extracting entities-based factual knowledge from $S_i$.
 By considering summarization as a knowledge filter, we can achieve information augmentation with greater efficiency.
 
\subsubsection{LLMs as Named Entity Elucidator}
\label{sec:LLMs as Named Entity Elucidator}
 The resolution of multi-span questions usually entails identifying named entities. This intricate task can be greatly improved by tapping into the specialized knowledge of named entities. This expertise directly aids the model in comprehending rare lexical items within its pre-training corpus and adapting to context-specific terminologies during fine-tuning. 
  
 While previous research has delved into entity knowledge either through training or direct application of fine-tuned NER models like spaCy and BERN2, these models have limitations stemming from their narrow training datasets and their tendency to solely provide extracted entities without contextualized meanings due to their inflexible design.  
 In comparison, generative large language models, endowed with advanced representation capabilities and abundant training data, present a promising avenue for achieving more precise and comprehensive entity delineations.
  
 GIANT utilizes a large language model to generate entity elucidation $E_i$ represented as $\{e_0,...,e_h\}$, from summary $S_i$ ,from summaries $S_i$ within each context $C_i$, where $i$ ranges from 1 to $n$. 
 Following this, it adopts a hybrid methodology integrating both regular expressions and semantic dependency analysis models, exemplified by spaCy, to parse the produced text.
 This systematic procedure culminates in the establishment of a ``Entity-Elucidation'' knowledge base oriented towards the elucidation of entities. 
 Furthermore, as \figref{fig:entity_insertion} presented, we seamlessly integrates the retrieved entity $e_t \in E_i$ mentioned within context $C_i$ by inserting its corresponding explanation immediately after the entity mention, significantly enriching the original text content while ensuring coherence and clarity are upheld.
 
\begin{figure*}[h]
	\centering
	\includegraphics[width=10cm]{EntityInsert.png}
	\caption{The Process of Inserting Entity Explanation into Context}
	\label{fig:entity_insertion}
\end{figure*}

\begin{figure*}[h]
	\centering
	\includegraphics[width=10cm]{RandomConcat.png}
	\caption{The Process of Concatenation of Original Context and Auxiliary Information}
	\label{fig:random_concatenate}
\end{figure*}  

\subsubsection{LLMs as Entity Relationship Extractor}
\label{sec:llm_as_relationship_extrator}
 In addition to entity elucidation, understanding entity relations can also assist the model in untangling intricate logical relationships within complex context, thereby facilitating the clarification of interconnected concepts focused on the question-and-answering process.
 With the robust language comprehension capabilities, contextual sensitivity, and extensive external knowledge, large language models can adeptly capture entity associations within extensive contents and transform this structured knowledge into natural language expressions, thus integrating factual knowledge into the model in textual form. 
 
 After extracting the textual entity relationships $R_i$ from each context $C_i$ provided in the MSQA benchmark, we first calculate the ratio $r_l$, which is defined as the quotient of the cumulative length of the generated text $R_i$ and the total length of the original context $C_i$. 
 As depicted in Figure 2, we subsequently partition the enhanced text $R_i$ and the original context $C_i$ into several segments, denoted as $\tilde{R_i} = \{\hat{r_1}, \hat{r_2}, ..., \hat{r_w}\}$ and $\tilde{C_i} = \{\hat{t_1}, \hat{t_2}, ..., \hat{t_w}\}$, respectively.
 After partitioning, the number of enhanced text segments $|\tilde{R_i}|$ equals the number of original context segments $|\tilde{C_i}|$, and the ratio of the length of each enhanced text segment $|t_i|$ to each original context segment $|c_i|$ is consistent with $r_l$. 
 Next, we maintain the order of the original context segments while shuffling the order of the enhanced text segments, represented as $R_i' = \{r_{\sigma(1)}, r_{\sigma(2)}, ..., r_{\sigma(m)}\}$, where $\sigma$ is a random permutation of the indices of the enhanced text segments. Then, we sequentially insert the shuffled enhanced text segments into the original text segments. This concatenation results in a new context $C' = \{c_1, r_{\sigma(1)}, c_2, r_{\sigma(2)}, ..., c_n, r_{\sigma(n)}\}$ enriched with relational knowledge.
	

\subsubsection{LLMs as Content Continuator}
\label{sec:llm_as_continuator}




\section{Experiments}
In this section, we compare our information augmentation approach with multiple strong baseline on multi-span question answering. We first introduce the datasets and experiment setup, then show the experimental results and analysis for different model.

\subsection{Evaluation Dataset}
\label{sec:datasets}
We conducted experiments on MultiSpanQA(Li et al., 2022), a recently introduced Reading Comprehension dataset designed for multi-span question answering. This dataset comprises 6.5K multi-span examples in which the questions represent user queries issued to the Google search engine, and the contexts are extracted from the English Wikipedia. It's worth noting that there is a expand variant of MultiSpanQA known as MultispanQA(expand), which intakes single-span and answerable questions. However, we did not perform a comparison with the expanded dataset due to its relatively lower proportion of multi-span QA pairs.

\subsection{Experimental Setup}
For all competing models and our model, we use the HuggingFace implementation of $\text{BERT}_{base}$ or $\text{RoBERTa}_{base}$ as the \textit{encoder} with $\textit{max\_len}$ = 512. We set the initial learning rate as $3 \times 10^{-5}$ and  $\textit{batch\_size}$ = 4, and use the BERTAdam optimizer with a weight decay of 0.01. Our approach does not involve tuning the parameters on the validation set. Instead, we rely on the model checkpoints obtained after 5 epochs. Next, we introduce the comparison model and evaluation metrics in our experiments.

\subsubsection{\textit{Model Under Comparison}}
\label{sec:baselines}
We introduce two comstracting models approaches to multi-span answer extraction : \textbf{TASE} (Segal et al., 2020) and \textbf{LIQUID}(Lee et al., 2023). TASE utilizes a tag-based span extraction model which identifies multi-span answers though the assigning a tag to every input token with BIO tagging scheme. On the other hand, LIQUID serves as a framework for generating multi-span QA datasets to improve model performance.

To enhance the context with auxiliary information, we employ two distinct   approaches: \textbf{$\text{AUG}_{c}$} and \textbf{$\text{AUG}_{eree}$}, where \textbf{AUG} is our automatic data augmentation framework,and the suffix indicates which kind of information is injected into the context. \textbf{$\text{AUG}_{C}$} enriches the context with continue writing, while \textbf{$\text{AUG}_{EREE}$} supplements context with entities information including explanation and relationship analysis.
Specifically, we leverage ChatGPT as a knowledge source to linearize the relevant information from large language models. in texts format and seamlessly integrate into the original contexts, thus reinforces the information of model inputs.


\subsubsection{\textit{Evaluation Metrics}}
\label{sec:metrics}
We use two automatic metrics for evaluation: Exact Match and Overlap F1 score.
\begin{itemize}
	\item \textbf{Exact Match}. An exact match occurs when a predicted span fully matches one of the ground-truth
	answer spans. We calculate the micro-average precision, recall and f1 score for the extract match
	metric.
	
	\item \textbf{Overlap F1 score}. Overlap F1 score is the macro-average f1 score, where the f1 score for each
	example is computed by treating the prediction and gold as a bag of tokens.
\end{itemize}


\subsection{Experimental Results and Analysis}
In this section, we compare $\text{AUG}$ with all competing models described above quantitatively.

\subsubsection{\textit{Comparison Results}}
We evaluate our model as well as baselines 
\(( Section ~\ref{sec:baselines} )\) on the development splits of multi-span datasets \(( Section  ~\ref{sec:datasets})\) using automatic metrics \(( Section ~\ref{sec:datasets})\). The comparison results are shown in Table \ref{tab:bertall}, Table \ref{tab:robertaall}.

\begin{table*}[width=\textwidth,cols=9,pos=h]  % 
	\caption{Approach performance on complete MultiSpanQA valid set based on $\text{BERT}_{base}$.} 
	\label{tab:bertall}
	\begin{tabular*}{\textwidth}{@{\extracolsep{\fill}}lccccccc}
		\toprule
		\multirow{2}{*}{\textbf{Model}} & \multicolumn{3}{c}{Exact Match} & \multicolumn{3}{c}{Partial Match}  \\
		\cline{2-7} 
		\addlinespace
		& F\((\%)\) & P\((\%)\) & R\((\%)\) & F\((\%)\) & P\((\%)\) & R\((\%)\) \\
		\midrule
		TASE   & 60.28 & 55.59 & 65.83 & 78.16 & 78.27 & 78.06 \\ 
		LIQUID & 61.44 & 58.39 & 64.84 & 78.56 & 78.65 & 78.46 \\
		$\text{AUG}_{c}$ & 63.05 & 58.51 & 68.34 & 79.42 & 78.70 & 80.14 \\
		$\text{AUG}_{eree}$  & 63.93 & 60.22 & 68.13 & 77.50 & 77.07 & 77.94 \\
		$\text{AUG}_{ereec}$ & 62.90  & 61.02 & 64.89 & 76.72 & 78.56 & 74.97 \\
		$\text{Bagging}_{eeerc}$ & 63.63  & 61.53 & 65.88 & 78.24 & 80.00 & 76.56 \\
		$\text{Bagging}_{eeerc+liquid}$ & 64.44 & 61.63 & 67.50 & 79.5 & 80.49 & 78.57 \\
		\bottomrule
	\end{tabular*}
	%	\noindent{\footnotesize{\textsuperscript{1} Tables may have a footer.}}
\end{table*}

\begin{table*}[width=\textwidth,cols=8,pos=h]  
	\caption{Approach performance on complete MultiSpanQA valid set based on $\text{RoBERTa}_{base}$.} 
	\label{tab:robertaall}
	\begin{tabular*}{\textwidth}{@{\extracolsep{\fill}}lccccccc}
		\toprule
		\multirow{2}{*}{\textbf{Model}} & \multicolumn{3}{c}{Exact Match} & \multicolumn{3}{c}{Partial Match}  \\
		\cline{2-7} 
		\addlinespace
		& F\((\%)\) & P\((\%)\) & R\((\%)\) & F\((\%)\) & P\((\%)\) & R\((\%)\) \\
		\midrule
		TASE & 68.00 & 65.06 & 71.22 & 83.13 & 83.05 & 83.22 \\ 
		LIQUID & 68.33 & 66.68 & 70.07 & 82.71 & 82.45 & 82.98 \\
		$\text{AUG}_{c}$ & 70.35 & 67.35 & 73.63 & 84.06 & 83.38 & 84.76 \\
		$\text{AUG}_{eree}$ & 69.15 & 67.81 & 70.54 & 82.85 & 83.90 & 81.83 \\
		$\text{AUG}_{ereec}$ & 70.44 & 67.83 & 73.26 & 82.80 & 82.45 & 83.15 \\
		$\text{Bagging}_{eeerc}$ & 70.48  & 69.11 & 71.90 & 84.28 & 85.61 & 83.00 \\
		$\text{Bagging}_{eeerc+liquid}$ & 70.86 & 69.03 & 72.79 & 84.82 & 85.53 & 84.12 \\
		\bottomrule
	\end{tabular*}      
	%	\noindent{\footnotesize{\textsuperscript{1} Tables may have a footer.}}
\end{table*}
Table~\ref{tab:bertall} and Table~\ref{tab:robertaall} illustrate the performance comparison between the proposed approaches, $\text{AUG}_{c}$ and $\text{AUG}_{eree}$, and several strong baselines, including the previous state-of-the-art model LIQUID. These comparisons are conducted using both the $\text{BERT}_{base}$ and $\text{RoBERTa}_{base}$ encoders, and regard multi-span question answering as a \textit{BIO} sequence tagging task to predict each token whether it is a part or begin of an answer. 
Notably,$\text{AUG}_{c}$ exhibit superior performance across the evaluate dataset on all metrics. However, on Partial Match scores, $\text{AUG}_{eree}$ demonstrates slightly lower performance compared to TASE, and especially lower than LIQUID when employing the $\text{BERT}_{base}$ encoder.
Importantly, the performance of $\text{AUG}_{c}$ consistently outperforms LIQUID and TASE on all metrics and encoders, irrespective of the encoder setting.
These results demonstrate the effectiveness of our proposed framework, as well as the efficacy of the information augmentation strategy.

To be more specific, Table~\ref{tab:bertall} shows comparisons of metrics among all competing models 100 backed by $\text{BERT}_{base}$. We can see that our proposed framework, AUG, consistently outperforms all other baselines across multispanQA dataset. Backed by $\text{BERT}_{base}$, $\text{AUG}_{c}$ achieves EM and Overlap F1 scores of 63.05 and 79.42, respectively. Moreover, when equipped with entities’ information,  $\text{AUG}_{eree}$ achieves even higher EM F1 scores of 63.93 but relatively lower Overlap F1 of 77.50 than baselines on the same encoder. These results showcase substantial improvements over the previous state-of-the-art model, LIQUID, with EM F1 score enhancements ranging from 1.61 to 2.49 percents across validation datasets. 

Additionally, when utilizing $\text{RoBERTa}_{base}$ instead of $\text{BERT}_{base}$, $\text{AUG}_{c}$ achieves EM and Overlap F1 scores of 70.35 and 84.06 respectively on the same datasets. These scores represent EM and Overlap F1 improvements of 2.02 and 0.93 compared to the previous setup. For  $\text{AUG}_{eree}$, the EM F1 score represents an enhancement of 0.82, with the EM and Overlap F1 values of 69.16 and 82.85 respectively. Notably, the partial metrics also indicate lower values compared to TASE and LIQUID, in line with the result supported by $\text{BERT}_{base}$. This is because augmenting the model with entity information, including definition and relationship knowledge, strengthens its ability to capture and understand entity concepts, which leads the model to prefer complete entity spans or empty span set as answers rather than partial entity span, and therefore a decrease in the partial recall and ultimately a lower partial F1 scores and a higher EM F1.


To further substantiate our explanation for the suboptimal performance of our model on the Overlap F1 metric, we conducted a detailed examination of the predictions made by $\text{AUG}_{eree}$ and two baseline models on the validation dataset. In essence, we tallied the instances where these models predicted empty answers and recalculated their Overlap F1 scores on non-empty predictions. This allowed us to investigate whether the $\text{AUG}_{eree}$ model aligns with our hypothesis, which posits that its extensive learning of entity knowledge during training makes it inclined to output either a complete and accurate answer span or no answer at all, as opposed to a partially correct answer span.

As presented in Table~\ref{tab:answer_counts}, $\text{AUG}_{eree}$ indeed predicted a higher number of empty answers compared to TASE and LIQUID, while achieving relatively higher Overlap F1 scores on non-empty predictions. Specifically, when equipped with $\text{BERT}_{base}$ as the encoder, $\text{AUG}_{eree}$ obtained an F1 score of 0.7976 on non-empty answer predictions, whereas TASE and LIQUID scored 0.7964 and 0.7909, respectively. With $\text{RoBERTa}_{base}$, $\text{AUG}_{eree}$ achieved an Overlap F1 score of 0.8414, surpassing TASE and LIQUID, which scored 0.8404 and 0.8357, respectively. Additionally, it is worth noting that $\text{AUG}_{eree}$ consistently predicted more empty answers, whether using BERT or RoBERTa.

These findings lend support to our conjecture that the introduction of entity knowledge leads to a slight reduction in the model's Overlap F1 scores. This suggests that utilizing LLM as a knowledge source to linearize entity information from LLM text and integrate it into the original context empowers the model to acquire greater entity knowledge, thereby exhibiting a preference for more accurate and complete answer spans, or simply providing no answer.



\begin{table}[htbp]
	\caption{The statistics of answers span predicted by $\text{AUG}{eree}$, $\text{AUG}{ereec}$ and baselines}
	\label{tab:answer_counts_vertical}
	\centering
	\begin{tabular}{lcc}
		\toprule
		\textbf{Category} & \textbf{BERT} & \textbf{RoBERTa} \\
		\midrule
		\multicolumn{3}{l}{\textbf{TASE}} \\
		\hspace{5mm} empty preds & 2.45 & 0.61 \\
		\hspace{5mm} non-empty pmf1 & 79.64 & 84.04 \\
		\multicolumn{3}{l}{\textbf{LIQUID)}} \\
		\hspace{5mm} empty preds & 2.14 & 1.53 \\
		\hspace{5mm} non-empty pmf1 & 79.09 & 83.57 \\
		\multicolumn{3}{l}{$\textbf{AUG}{eree}$ } \\
		\hspace{5mm} empty preds & \textbf{6.43} & \textbf{2.91} \\
		\hspace{5mm} non-empty pmf1 & \textbf{79.76} & \textbf{84.14} \\
		\multicolumn{3}{l}{$\textbf{AUG}{ereec}$ } \\
		\hspace{5mm} empty preds & \textbf{7.96} & \textbf{3.22} \\
		\hspace{5mm} non-empty pmf1 & \textbf{80.20} & \textbf{83.78} \\
		\bottomrule
	\end{tabular}
\end{table}



Totally, the result, displayed in Table~\ref{tab:robertaall} demonstrates the same trends to Table~\ref{tab:bertall}. And the outcome highlights robustness in effectively generalizing across different datasets without requiring hyperparameter re-tuning. 

\subsubsection{\textit{Discussion}}
Table~\ref{tab:bertall} and Table~\ref{tab:robertaall} discuss the performance of different augmentation integrated strategies, including the results-bagging methods whose outputs are voting results of $\text{AUG}_{c}$, $\text{AUG}_{eree}$ as well as LIQUID, and the input-fusion model $\text{AUG}_{ceree}$ who injects all kinds of information above into input contexts.

In detail, with EM f1 scores of 62.90 and 64.44 in Table~\ref{tab:bertall} and 70.44 and 70.86 in Table~\ref{tab:robertaall}, both the $\text{AUG}_{ceree}$ and Bagging methods consistently surpass TASE and LIQUID, which exhibits robust effectiveness of information injection strategies. However, there is a little decrease caused by fusing all auxiliary information when contrast with single information augmentation strategies and the bagging method. This may be due to the likelihood that incorporating all of the augmentation information into the model inputs will confuse the model by introducing excessive auxiliary knowledge and underrepresented original context proportion. Therefore it may be more useful that adding limit information into context, and using result-bagging method, a multi-model voting to bringing all information into model with an indirect way.

From the Partial Match perspective, $\text{AUG}_{ceree}$ and the Bagging method achieve 76.72 and 79.52 in Table~\ref{tab:bertall}, and 82.80 and 84.82 in Table~\ref{tab:robertaall}. In accordance with Exact Match metrics, Bagging demonstrate a overall superior performance. Meanwhile overlap f1 score of $\text{AUG}_{eree}$ is inferior to TASE but superior to LIQUID ,with relatively higher precision and relativelv higher recall, which  is comparable to all single augmented models such as $\text{AUG}_{eree}$. And its weak performance on overlap f1 also reveals complete entities preference of this information injection approach.

Furthermore, we stratified the data within MultispanQA according to answer types, specifically categorizing them into DESC, NUM, and ENTYS. We subsequently conducted a comparative analysis of model performance within each of these subcategories. In particular, the results are presented in Tables~\ref{tab:bertsub} and Table~\ref{tab:robertasub}, supported by $\text{BERT}_{base}$ and $\text{RoBERTa}_{base}$, respectively.
As indicated in Tables~\ref{tab:bertsub} and~\ref{tab:robertasub}, our proposed models exhibit superior performance in terms of EM F1 scores for all categorizing. However, they demonstrate suboptimal performance in terms of overlap F1 scores.



\begin{table*}[ht]
	\caption{Model performance on complete MultiSpanQA valid Subset with different answer types based on $\text{BERT}_{base}$.}
	\label{tab:bertsub}
	\begin{tabular*}{\textwidth}{@{\extracolsep{\fill}}lccccccc}
		\toprule
		\multirow{2}{*}{\textbf{Type}} & \multirow{2}{*}{\textbf{Model}} & \multicolumn{3}{c}{Exact Match} & \multicolumn{3}{c}{Partial Match} \\
		\cmidrule{3-8} 
		& & F\((\%)\) & P\((\%)\) & R\((\%)\) & F\((\%)\) & P\((\%)\) & R\((\%)\) \\
		\midrule
		\multirow{6}{*}{DESC} & TASE & 32.76 & 27.33 & 40.87 & 64.46 & 68.61 & 60.79 \\ 
		& LIQUID & 36.69 & 31.10 & 44.71 & 66.05 & 65.95 & 66.15 \\
		& $\text{AUG}_{c}$ & 38.74 & 32.89 & 47.12 & 68.31 & 70.32 & 66.42 \\
		& $\text{AUG}_{eree}$ & 44.69 & 40.71 & 49.52 & 63.12 & 67.16 & 59.53 \\
		& $\text{AUG}_{ereec}$ & 40.63 & 38.30 & 43.27 & 64.25 & 71.15 & 58.57 \\
		& $\text{Bagging}_{eeerc}$ & 39.19  & 36.86 & 41.83 & 64.46 & 75.12 & 56.45 \\
		& $\text{Bagging}_{eeerc+liquid}$  & 39.91 & 36.69 & 43.75 & 66.70 & 73.68 & 60.93 \\
		\midrule
		\multirow{6}{*}{NUM} & TASE & 33.33 & 30.23 & 37.14 & 60.98 & 71.99 & 52.89 \\ 
		& LIQUID & 34.33 & 31.25 & 38.10 & 60.68 & 68.15 & 54.69 \\
		& $\text{AUG}_{c}$ & 35.96 & 33.33 & 39.05 & 64.47 & 71.56 & 58.65 \\
		& $\text{AUG}_{eree}$ & 36.20 & 34.48 & 38.10 & 59.02 & 65.63 & 53.62 \\
		& $\text{AUG}_{ereec}$ & 36.71 & 37.25 & 36.19 & 57.09 & 69.04 & 48.67 \\
		& $\text{Bagging}_{eeerc}$ & 35.58 & 35.92 & 35.24 & 58.62  & 68.09 & 51.46 \\
		& $\text{Bagging}_{eeerc+liquid}$  & 35.94 & 34.82 & 37.14 & 60.73 & 71.15 & 52.97 \\
		\midrule
		\multirow{6}{*}{ENTYS} & TASE & 66.30 & 62.21 & 70.96 & 81.16 & 80.37 & 81.96 \\ 
		& LIQUID & 67.17 & 65.25 & 69.21 & 81.66 & 81.69 & 81.63 \\
		& $\text{AUG}_{c}$ & 68.47 & 64.44 & 73.03 & 81.93 & 80.57 & 83.34 \\
		& $\text{AUG}_{eree}$ & 68.36 & 64.64 & 72.53 & 80.55 & 79.21 & 81.93 \\
		& $\text{AUG}_{ereec}$ & 67.54 & 65.60 & 69.59 & 79.49 & 80.16 & 78.83 \\
		& $\text{Bagging}_{eeerc}$ & 68.68  & 66.49 & 71.03 & 81.11 & 81.39 & 80.83 \\
		& $\text{Bagging}_{eeerc+liquid}$ & 69.65 & 66.94 & 72.59 & 82.31 & 82.07 & 82.55 \\
		\bottomrule
	\end{tabular*}
\end{table*}

\begin{table*}[ht]
	\caption{Model performance on complete MultiSpanQA valid Subset with different answer types based on $\text{RoBERTa}_{base}$.}
	\label{tab:robertasub}
	\begin{tabular*}{\textwidth}{@{\extracolsep{\fill}}lccccccc}
		\toprule
		\multirow{2}{*}{\textbf{Type}} & \multirow{2}{*}{\textbf{Model}} & \multicolumn{3}{c}{Exact Match} & \multicolumn{3}{c}{Partial Match} \\
		\cmidrule{3-8} 
		& & F\((\%)\) & P\((\%)\) & R\((\%)\) & F\((\%)\) & P\((\%)\) & R\((\%)\) \\
		\midrule
		\multirow{6}{*}{DESC} & TASE & 45.53 & 40.84 & 51.44 & 75.44 & 76.61 & 74.31 \\ 
		& LIQUID & 48.68 & 44.76 & 53.37 & 73.27 & 71.57 & 75.04 \\
		& $\text{AUG}_{c}$ & 49.02 & 44.66 & 54.33 & 76.33 & 77.85 & 74.86 \\
		& $\text{AUG}_{eree}$ & 50.99 & 46.96 & 55.77 & 76.87 & 78.07 & 75.71 \\
		& $\text{AUG}_{ereec}$ & 49.24 & 44.71 & 54.81 & 71.87 & 73.03 & 70.74 \\
		& $\text{Bagging}_{eeerc}$ & 51.54 & 47.56 & 56.25 & 78.20 & 80.04 & 76.44 \\
		& $\text{Bagging}_{eeerc+liquid}$ & 47.70 & 43.78 & 52.40 & 77.43 & 80.63 & 74.47 \\
		\midrule
		\multirow{6}{*}{NUM} & TASE & 46.23 & 45.79 & 46.67 & 71.89 & 78.42 & 66.36 \\ 
		& LIQUID & 40.19 & 39.45 & 40.95 & 67.56 & 72.28 & 63.42 \\
		& $\text{AUG}_{c}$ & 50.69 & 49.11 & 52.38 & 72.86 & 80.22 & 66.74 \\
		& $\text{AUG}_{eree}$ & 42.40 & 41.07 & 43.81 & 66.67 & 73.51 & 61.00 \\
		& $\text{AUG}_{ereec}$ & 45.58 & 44.55 & 46.67 & 65.86 & 73.19 & 59.87 \\
		& $\text{Bagging}_{eeerc}$ & 43.32 & 41.96 & 44.76 & 68.57 & 76.54 & 62.11 \\
		& $\text{Bagging}_{eeerc+liquid}$ & 44.65 & 43.64 & 45.71 & 70.11 & 78.49 & 63.35 \\
		\midrule
		\multirow{6}{*}{ENTYS} & TASE & 72.57 & 69.94 & 75.41 & 84.90 & 84.32 & 85.48 \\ 
		& LIQUID & 72.95 & 71.77 & 74.16 & 85.02 & 84.75 & 85.29 \\
		& $\text{AUG}_{c}$ & 74.59 & 71.87 & 77.53 & 85.79 & 84.40 & 87.23 \\
		& $\text{AUG}_{eree}$ & 73.50 & 72.81 & 74.22 & 84.74 & 85.49 & 83.99 \\
		& $\text{AUG}_{ereec}$ & 75.04 & 72.81 & 77.41 & 85.37 & 84.46 & 86.29 \\
		& $\text{Bagging}_{eeerc}$ & 74.97 & 74.23 & 75.72 & 86.14 & 87.07 & 87.06 \\
		&$\text{Bagging}_{eeerc+liquid}$ & 75.85 & 74.52 & 77.22 & 86.73 & 86.73 & 86.74 \\
		\bottomrule
	\end{tabular*}
\end{table*}

\subsubsection{\textit{Ablation Experiment}}
At the end of this section, we conducted ablations on our approach to confirm the effectiveness of selecting the information injection proportion. For each QA data, we randomly split the original context and the auxiliary text, then concatenated them into a final augmented context with a specific proportion to ensure that the new input length meets $\textit{max\_len}$, which has a crucial impact on our approach. In practice, we determined the final text splicing ratio by calculating the ratio of the average length of the source text to the added information, which for the $\text{AUG}_{c}$ is 0.86.

Specifically, Table~\ref{tab:different_ratio_bert} and Table~\ref{tab:different_ratio_roberta} displays $\text{AUG}_{c}$'s performance with differential proportion to concatenate original contexts and continuation, on complete MultispanQA valid set,backed by $\text{BERT}_{base}$ and $\text{RoBERTa}_{base}$ respectively. We choose five proportions for information integration, which determine how much auxiliary information would be inject into each overflowed text segment. The results presented in tables indicate that using the ratio of their average lengths as the proportion of the overflow text composed of original text and auxiliary information is an effective approach. In detail, $\text{AUG}_{c}$,equipping with $\text{BERT}_{base}$, achieves an Exact Match F1 scores improvements of at least 4.57 compared to other proportions and an overlap F1 scores improvements of at least 2.07. In line with Table~\ref{tab:different_ratio_bert}, when $\text{AUG}_{c}$ equips with $\text{Roberta}_{base}$, it achieves an improvement of Exact Match F1 scores of 0.55 but an decrease of Overlap F1 scores of 0.17.



\begin{table*}[H] 
	\caption{Ablations of $\text{AUG}_{c}$ on different proportion for information concatenation, based on $\text{BERT}_{base}$.} 
	\label{tab:different_ratio_bert}
	\newcolumntype{C}{>{\centering\arraybackslash}X}
	\begin{tabular*}{\textwidth}{@{\extracolsep{\fill}}lccccccc}
		\toprule
		\multirow{2}{*}{\textbf{Proportion}} & \multicolumn{3}{c}{Exact Match} & \multicolumn{3}{c}{Partial Match}  \\
		\cline{2-7} 
		\addlinespace
		& F\((\%)\) & P\((\%)\) & R\((\%)\) & F\((\%)\) & P\((\%)\) & R\((\%)\) \\
		\midrule
		0.90 & 58.55 & 57.69 & 59.44 & 71.88 & 74.30 & 69.61 \\ 
		0.70 & 61.70 & 59.03 & 64.63 & 76.58 & 77.84 & 75.36 \\
		0.50 & 61.22 & 57.37 & 65.62 & 77.38 & 77.61 & 77.16 \\
		0.40 & 61.36 & 56.50 & 67.13 & 78.26 & 77.57 & 78.96 \\
		0.10 & 61.41 & 56.36 & 67.45 & 78.94 & 78.45 & 79.44 \\
		0.14 & 63.05 & 58.51 & 68.34 & 79.42 & 78.70 & 80.14 \\
		\bottomrule
	\end{tabular*}      
	%	\noindent{\footnotesize{\textsuperscript{1} Tables may have a footer.}}
\end{table*}



\begin{table*}[H] 
	\caption{Ablations of $\text{AUG}_{c}$ on different proportion for information concatenation, based on $\text{RoBERTa}_{base}$.} 
	\label{tab:different_ratio_roberta}
	\newcolumntype{C}{>{\centering\arraybackslash}X}
	\begin{tabular*}{\textwidth}{@{\extracolsep{\fill}}lccccccc}
		\toprule
		\multirow{2}{*}{\textbf{Proportion}} & \multicolumn{3}{c}{Exact Match} & \multicolumn{3}{c}{Partial Match}  \\
		\cline{2-7} 
		\addlinespace
		& F\((\%)\) & P\((\%)\) & R\((\%)\) & F\((\%)\) & P\((\%)\) & R\((\%)\) \\
		\midrule
		0.90 & 58.13 & 56.92 & 59.39 & 71.70 & 73.49 & 69.98 \\ 
		0.70 & 66.97 & 65.42 & 68.60 & 80.05 & 80.80 & 79.32 \\
		0.50 & 68.94 & 67.23 & 70.75 & 82.43 & 83.23 & 81.64 \\
		0.40 & 68.95 & 66.11 & 71.06 & 83.71 & 83.71 & 81.71 \\
		0.10 & 69.80 & 66.47 & 73.47 & 84.23 & 83.49 & 84.99 \\
		0.86 & 70.35 & 67.35 & 73.63 & 84.06 & 83.38 & 84.76 \\
		\bottomrule
	\end{tabular*}      
	%	\noindent{\footnotesize{\textsuperscript{1} Tables may have a footer.}}
\end{table*}



%\begin{table}[H] 
%	\caption{Ablations of $\text{AUG}_{c}$ on different proportion for information concatenation, based on $\text{BERT}_{base}$.} 
%	\label{tab:different_ratio_roberta}
%	\newcolumntype{C}{>{\centering\arraybackslash}X}
%	\begin{tabularx}{\textwidth}{p{1.7cm}CCCCCCC}
	%		\toprule
	%		\multirow{2}{*}{\textbf{Proportion}} & \multicolumn{3}{c}{Exact Match} & \multicolumn{3}{c}{Partial Match}  \\
	%		\cline{2-7} 
	%		\addlinespace
	%		& F\((\%)\) & P\((\%)\) & R\((\%)\) & F\((\%)\) & P\((\%)\) & R\((\%)\) \\
	%		\midrule
	%		0.90 & 58.13 & 56.92 & 59.39 & 71.70 & 73.49 & 69.98 \\ 
	%		0.70 & 66.97 & 65.42 & 68.60 & 80.05 & 80.80 & 79.32 \\
	%		0.50 & 68.94 & 67.23 & 70.75 & 82.43 & 83.23 & 81.64 \\
	%		0.40 & 68.95 & 66.11 & 71.06 & 83.71 & 83.71 & 81.71 \\
	%		0.10 & 69.80 & 66.47 & 73.47 & 84.23 & 83.49 & 84.99 \\
	%		0.86 & 70.35 & 67.35 & 73.63 & 84.06 & 83.38 & 84.76 \\
	%		\bottomrule
	%	\end{tabularx}      
%	%	\noindent{\footnotesize{\textsuperscript{1} Tables may have a footer.}}
%\end{table}
%
\input{conclusion}

\vspace{6pt} 

%%%%%%%%%%%%%%%%%%%%%%%%%%%%%%%%%%%%%%%%%%
%% optional
%\supplementary{The following supporting information can be downloaded at:  \linksupplementary{s1}, Figure S1: title; Table S1: title; Video S1: title.}

% Only for journal Methods and Protocols:
% If you wish to submit a video article, please do so with any other supplementary material.
% \supplementary{The following supporting information can be downloaded at: \linksupplementary{s1}, Figure S1: title; Table S1: title; Video S1: title. A supporting video article is available at doi: link.}

% Only for journal Hardware:
% If you wish to submit a video article, please do so with any other supplementary material.
% \supplementary{The following supporting information can be downloaded at: \linksupplementary{s1}, Figure S1: title; Table S1: title; Video S1: title.\vspace{6pt}\\
%\begin{tabularx}{\textwidth}{lll}
%\toprule
%\textbf{Name} & \textbf{Type} & \textbf{Description} \\
%\midrule
%S1 & Python script (.py) & Script of python source code used in XX \\
%S2 & Text (.txt) & Script of modelling code used to make Figure X \\
%S3 & Text (.txt) & Raw data from experiment X \\
%S4 & Video (.mp4) & Video demonstrating the hardware in use \\
%... & ... & ... \\
%\bottomrule
%\end{tabularx}
%}

%%%%%%%%%%%%%%%%%%%%%%%%%%%%%%%%%%%%%%%%%%
%\authorcontributions{For research articles with several authors, a short paragraph specifying their individual contributions must be provided. The following statements should be used ``Conceptualization, X.X. and Y.Y.; methodology, X.X.; software, X.X.; validation, X.X., Y.Y. and Z.Z.; formal analysis, X.X.; investigation, X.X.; resources, X.X.; data curation, X.X.; writing---original draft preparation, X.X.; writing---review and editing, X.X.; visualization, X.X.; supervision, X.X.; project administration, X.X.; funding acquisition, Y.Y. All authors have read and agreed to the published version of the manuscript.'', please turn to the  \href{http://img.mdpi.org/data/contributor-role-instruction.pdf}{CRediT taxonomy} for the term explanation. Authorship must be limited to those who have contributed substantially to the work~reported.}

%=====================================
% References, variant A: external bibliography
%=====================================
%\bibliography{your_external_BibTeX_file}

\section*{Acknowledgements}
This research was supported by the Natural Science Foundation of Zhejiang Province, China (Grant No. LQ22F020027), Fundamental Research Funds of Zhejiang Sci-Tech University (Grant No. 23232138-Y), Liaoning Provincial Natural Science Foundation of China (Grant No. 2022-KF-21-01) and the Key Research and Development Program of Zhejiang Province, China (Grant No. 2023C01041 and 2022C01079).

\bibliographystyle{cas-model2-names}
% 
\bibliography{refs}
%\end{multicols}
\end{document}

